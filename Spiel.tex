\documentclass[10pt,a4paper]{article}

\usepackage{geometry}
\geometry{a4paper, top=30mm, left=30mm, right=30mm, bottom=30mm,
headsep=10mm, footskip=12mm}
\usepackage[utf8]{inputenc}
\usepackage[ngerman]{babel}
\usepackage[T1]{fontenc}
\usepackage{lmodern}
\usepackage{graphicx}
\usepackage{wrapfig}
\usepackage{pdfpages}
\usepackage{longtable}
\usepackage{enumerate}
\usepackage{enumitem}
\usepackage{wasysym}
\usepackage{amssymb}




\usepackage[footsepline,headsepline]{scrpage2}
\pagestyle{scrheadings}
\ohead{\thetitle}
\ihead{Julian Petruck und Leona Goebbels}
\ofoot{Seite \thepage}
\cfoot{} % if left empty, removes page number at center
\title{Projekt}
\author{Julian Petruck und Leona Goebbels}



\begin{document}
\maketitle
\newcommand{\thetitle} {Projekt}

\section*{Grundkonzept}
\begin{itemize}
\item Name: Quälgeist
\item Art: Textadventure
\item Zusammenfassung:\\
Erweiterung/ Abwandlung vom Brettspiel Cluedo.\\
Kind (Quälgeist) hat Mord an der Putzfrau beobachtet und Du als Ermittler musst den Mord aufklären.\\
Kind hat Hinweise, will diesen aber nicht teilen. Um Hinweis zu bekommen muss man Aufgaben erledigen, die das Kind (der Quälgeist) einem stellt. Das Kind ist sehr launisch und entscheidet sich oft um, nun doch nicht zu helfen oder gibt falsche Tipps.\\
Wenn Du eine Vermutung hast, dann kannst Du diese äußern. 
Wenn Du aber dreimal eine falsche Verdächtigung gemacht hast, dann verlierst Du deinen Job und auch das Spiel.\\
\ \\
Mögliche Täter:
\begin{itemize}
\item Vater
\item Mutter
\item Gärtner
\item Koch
\item Nachbar
\item Besuch
\end{itemize}
Mögliche Tatorte:
\begin{itemize}
\item Eingangsbereich
\item Schlafzimmer
\item Küche
\item Garten
\item Wohnzimmer
\item Arbeitszimmer
\end{itemize}
Mögliche Tatwaffen:
\begin{itemize}
\item Pistole
\item Messer
\item Seil
\item Spaten
\item Gift
\item Pokal
\end{itemize}
\item Pseudonym: jupleg
\item Alleinstellungsmerkmal:
\begin{itemize}
\item Minispiele in Hauptspiel eingebaut
\item Eingabe mit Enter (Satzzeichen am Ende oder in Satzmitte möglich)
\item Viele verschiedene, randomisierte Ausgaben auf nicht verstandene Eingaben
\end{itemize}
\end{itemize}

\section*{Nutzertest}
\subsection*{Zielsetzung}
Bei unserem Nutzertest hat uns interessiert, welche Eingaben der Nutzer tätigt (die wir nicht bedacht haben) und ob er unseren Aufbau mit den verschiedenen Situationen in denen nicht immer alle Fragen gehen verstanden hat. 
\subsection*{Aufgabenstellung}
Wir haben dem Nutzer jetzt keine feste Aufgabe (z.B. 'gehe ins Arbeitszimmer und sehe dich dort um') gegeben, da wir schauen wollten, ob dem Nutzer alle Möglichkeiten klar sind. Die Aufgabenstellung war lediglich: 'Spiele einfach!'. 
\subsection*{Auswertung}
Aufgetretene Probleme (nach Schwere sortiert, dabei ist das oberste das Schwerstwiegendste):
\begin{itemize}
\item
\item
\end{itemize}

\section*{Ausblick}
\begin{itemize}
\item Kindliche Sprache besser ausnutzen (mehr Witze), wir haben das nicht umgesetzt, da dies keine programmiertechnische Herausforderung ist und daher keine hohe Priorität bekommen hat
\item zu Beginn wird ein Schriftzug eingeblendet, "Quaelgeist", könnte man noch ohne ae sondern mit ä machen, sieht dann aber nicht mehr ganz so schön aus, daher haben wir uns dagegen entschieden
\end{itemize}
\end{document}

